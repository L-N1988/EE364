\documentclass[12pt]{article}
\usepackage{fullpage,graphicx,psfrag,amsmath,amsfonts,amsthm,verbatim}
\usepackage[small,bf]{caption}
\usepackage{graphicx}
\usepackage{probsoln}

\input defs.tex

\title{HW1: Convex sets}
\author{Matrix}

\begin{document}
\maketitle

Homework 1, due Friday 7/1/22:  2.9, 2.12a-e, 2.15, 2.4, A1.4, A2.7, 2.13.

\begin{solution}[2.9(a)]

	\textit{Voronoi region} definition yields
    \begin{align*}
			\|x-x_0\|_2\le \|x-x_i\| & \Longleftrightarrow (x-x_0)^T(x-x_0) \le (x-x_i)^T(x-x_i) \\
															 &\Longleftrightarrow x^Tx-2xx_0^T+x_0^Tx_0\le x^Tx-2xx_{i}^T+x_{i}^Tx_{i}\\
															 &\Longleftrightarrow (x_i-x_0)^Tx \le \frac{1}{2}(x_i-x_0)^T(x_i+x_0)
    .\end{align*}
		
		Note that the result above defines a halfspace for each $i$. Thus, we can express  $V$ in the form  $V=\{ x \mid Ax \preceq b\}$ with
		\begin{align*}
			A = \begin{bmatrix} (x_{1}-x_0)^T \\ \vdots \\ (x_{K}-x_0)^T \end{bmatrix} ,\ b=\begin{bmatrix} \frac{1}{2}(x_1-x_0)^T(x_1+x_0) \\ \vdots \\ \frac{1}{2}(x_K-x_0)^T(x_K-x_0) \end{bmatrix} 
		.\end{align*}

\end{solution}

\begin{solution}[2.9(b)]

	Since polyhedron $P$ has nonempty interior, we can express $P$ in the form $P=\{ x \mid Ax \preceq b\}$\footnote{If $P$ only contains hyperplane, then $P$ has empty interior, contradicting the assumption. This form contains both hyperplanes and halfspaces.} with $A \in \reals^{K\times n} $ and $b\in \reals^{K} $.
	We can choose any point $x_0$ from $P$'s interior, then take a mirro image of $x_0$ with respect to a hyperplane $\{a_i^T x = b_i\}$ with $a_{i} = (x_{i}-x_0)$ and $b_{i} = \frac{1}{2}(x_i-x_0)^T(x_i-x_0)$ to get $x_{i}$.
	Thus, any point $x\in P$ has shorter (or equal, when on hyperplane) distance to $x_0$ than $x_{i}$.

	The mirro image $x_{i}$ of $x_0$ with respect to hyperplane $\{a_i^T x = b_i\}$ satisfies
	\begin{align*}
		\begin{cases}
		\frac{\|a_{i}^{T} x_0-b_{i}\|}{\|a_{i}\|} =	\frac{\|a_{i}^{T} x_i-b_{i}\|}{\|a_{i}\|} \Longleftrightarrow a_{i}^{T} x_0-b_{i} = -1\cdot(a_{i}^{T} x_i-b_{i}), \\
x_{i} = x_0 + \lambda a_{i}.
		\end{cases}
	\end{align*}

	Solving $\lambda$ yields:
	\begin{align*}
		\lambda = \frac{2(b_{i}-a_{i}^Tx_0)}{\|a_{i}\|^2}
	.\end{align*}
	Thus, we can choose  
	\begin{align*}
		x_{i} = x_0 + \frac{2(b_{i}-a_{i}^Tx_0)}{\|a_{i}\|^2}a_{i},\ i = 1, \ldots, K
	\end{align*}
	so that the polyhedron $P$ is the \textit{Voronoi region} of $x_0$ with respect to $x_1,\ldots, x_{K}$.

\end{solution}

\begin{solution}[2.9(c)]

	% Wrong solution
	% Suppose each polyhedron $P^{(i)}=\{x  \mid A^{(i)}x\preceq b^{(i)}\}$ with $A^{(i)}\in \reals^{K\times n} $ and $b^{(i)}\in \reals^n$, we can choose any point $x_{1} \in P^{(1)} $ and take mirro image(s) of $x_1$ with respect to hyperplane(s) of $P^{(1)} $.
	% Then we use each image $x_{i}\in P^{(i)} $ to repeat this taking mirro image process until every polyhedron has image point in it. The image points set $\{x_1, \ldots, x_{m}\}$ describes polyhedron decomposition of $\reals^n$ as the \textit{Voronoi regions} generated by itself.
	
\end{solution}

\end{document}
